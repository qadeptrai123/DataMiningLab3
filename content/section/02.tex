\section{Thuật toán \textsc{Fast-Estimation}}

Mặc dù thuật toán Fast-Sampling có thời gian chạy tuyến tính trong khi vẫn duy trì các đảm bảo về mặt xấp xỉ, nhưng vẫn có $O(\log(kd))$ khi thực hiện chặn hội tụ xác suất, có thể ảnh hưởng trong thực tế của thuật toán khi xử lý các tập dữ liệu quy mô cực lớn. Để giải quyết vấn đề này, trong phần này, tác giả đề xuất một thuật toán dựa trên lấy mẫu nhanh hơn mang tên Fast-Estimation. Thuật toán Fast-Estimation có thể xấp xỉ hiệu quả tọa độ của từng cụm dự đoán trong thời gian chạy tuyến tính, với một sự đánh đổi nhỏ trong các đảm bảo về chất lượng phân cụm.

Ý tưởng chính: trước tiên tạo ra các tọa độ ứng viên có khả năng xấp xỉ chặt chẽ tọa độ của các tâm tối ưu. Sau đó, trong mỗi chiều của từng cụm dự đoán, một bộ ước lượng (estimator) được xây dựng bằng cách lấy mẫu theo phân phối đều. Bộ ước lượng này được thiết kế để cung cấp các ước tính chi phí phân cụm chính xác cho các tập con tọa độ có kích thước $(1-\alpha)m_i$. Cụ thể, đối với mỗi chiều của từng cụm dự đoán, bộ ước lượng được xây dựng bằng cách chọn ngẫu nhiên một tập $S_{ij}$ từ $P_{ij}$. Mỗi tọa độ được lấy mẫu sau đó được gán một trọng số bằng nhau, vì vậy xấp xỉ chi phí phân cụm thông qua các mẫu trọng số thay vì tính toàn bộ cụm dự đoán. Với các bộ ước lượng đã xây dựng, việc tìm kiếm tập hợp các tọa độ có chi phí phân cụm tối thiểu có thể được thực hiện trong thời gian hạ tuyến tính (sub-linear), loại bỏ nhân với $O(\log(kd))$ khỏi thời gian chạy của thuật toán Fast-Sampling.

\begin{algorithm}[H]
\caption{Fast-Estimation}
\label{alg:fast_estimation}
\begin{algorithmic}[1]
\Require Một bài toán $k$-means $(P, k, d)$, một tập các phân vùng $(P_1, P_2, ..., P_k)$ với tỷ lệ lỗi $\alpha$, và tham số $0 < \epsilon < 0.5$.
\Ensure Một tập $C \subset \mathbb{R}^d$ các tâm với $|C| = k$.
\For{$i \in [k]$}
    \For{$j \in [d]$}
        \State Lấy mẫu ngẫu nhiên và độc lập một tập $U_{ij}$ từ $P_{ij}$ với kích thước $O(\log(kd))$, sau đó khởi tạo $U'_{ij} = \emptyset$ và $\epsilon_1 = \frac{\epsilon}{126}$.
        \For{$q = 0$ to $O(\log(m\Delta^2_{max}))$}
            \State $l_{ij} = \sqrt{\frac{2^{q-1}}{(1-\alpha)m_i}}$.
            \For{$u \in U_{ij}$}
                \State $s(u) = \{ u + \epsilon_2\lambda l_{ij} : \lambda \in [-\frac{1}{\epsilon_2}, \frac{1}{\epsilon_2}] \cap \mathbb{Z} \}$, với $\epsilon_2 = \sqrt{\frac{\epsilon_1}{32}}$.
                \State $U'_{ij} = U'_{ij} \cup s(u)$.
            \EndFor
        \EndFor
        \State Lấy mẫu ngẫu nhiên và độc lập một tập $S_{ij}$ từ $P_{ij}$ với kích thước $O \left( \frac{\log(m^3d \log^3(m\Delta^2_{max})/\epsilon^2_1) \log(m\Delta^2_{max})}{\alpha\epsilon^4_1} \right)$, gán cho mỗi điểm trong $S_{ij}$ một trọng số $\frac{m_i}{|S_{ij}|}$.
        \State Xây dựng bộ ước lượng $\omega$ sao cho $\forall u \in U'_{ij}$, $\omega(u) = \sum_{p \in S_{ij} \setminus F(u)} \frac{m_i}{|S_{ij}|} \delta^2(p, u)$, trong đó $F(u)$ là tập hợp $(1 + 3\epsilon_1)\alpha|S_{ij}|$ điểm xa $u$ nhất trong $S_{ij}$.
        \State $c_{ij} = \arg \min_{u \in U'_{ij}} \omega(u)$.
        \State Gọi $I_{ij}$ là tập hợp $(1 - 2\alpha - \alpha\epsilon)m_i$ tọa độ gần $c_{ij}$ nhất từ $P_{ij}$.
    \EndFor
    \State $\hat{c}_i = (I_{ij})_{j \in [d]}$.
\EndFor
\State \Return $\{\hat{c}_1, \hat{c}_2, ..., \hat{c}_k\}$.
\end{algorithmic}
\end{algorithm}

\textbf{Phân tích thuật toán:}

% TODO PB: why Không mất tính tổng quát. thay giả định = giả sử 

Trong bước 3, đối với mỗi chiều của cụm dự đoán, thuật toán chọn một mẫu ngẫu nhiên $U_{ij}$ để xấp xỉ tọa độ của các tâm tối ưu. Theo Bổ đề \ref{lemma:4}, với xác suất hằng số, tồn tại ít nhất một tọa độ được lấy mẫu $u \in U_{ij}$ sao cho $\delta(u, Q'_{ij}) \leq \sqrt{2\delta^2(Q'_{ij}, \overline{Q'_{ij}})/|Q'_{ij}|}$. Sau đó, từ bước 4 liệt kê tất cả các độ dài khoảng ứng viên để xây dựng tập hợp các tọa độ ứng viên. Không mất tính tổng quát, có thể giả sử khoảng cách cặp tối thiểu giữa các tọa độ trong $P_{ij}$ là 1 và khoảng cách cặp tối đa là $\Delta_{max}$. Do đó, trong bước 5, tồn tại ít nhất một lần đoán $q$ cho độ dài thoả $\sqrt{\frac{2\delta^2(Q'_{ij}, \overline{Q'_{ij})}}{(1-\alpha)m_i}} \leq l_{ij} \leq \sqrt{\frac{4\delta^2(Q'_{ij}, \overline{Q'_{ij}})}{(1-\alpha)m_i}}$. Tiếp theo, trong các bước 7-8, theo Bổ đề \ref{lemma:5}, cũng tồn tại ít nhất một tọa độ $u' \in U'_{ij}$ sao cho $u'$ đủ gần với trọng tâm của $Q'_{ij}$, tức là $\delta(u', Q'_{ij}) \leq \sqrt{\epsilon_1\delta^2(Q'_{ij}, \overline{Q'_{ij}})/|Q'_{ij}|}$.

Đối với mỗi $u \in U'_{ij}$, gọi $\mathcal{N}_{ij}(u)$ là tập hợp $(1-\alpha)m_i$ tọa độ gần nhất từ $P_{ij}$ đến $u$. Gọi $O(u) = P_{ij} \setminus \mathcal{N}_{ij}(u)$ là tập hợp $\alpha m_i$ tọa độ xa nhất từ $P_{ij}$ đến $u$. Trước khi xây dựng bộ ước lượng $\omega$ (bước 9-10), tác giả bắt đầu bằng cách chia $\mathcal{N}_{ij}(u)$ thành $\gamma = \frac{(1+\epsilon_1) \log(m\Delta_{max}^2)}{\epsilon_1}$ khối. Cụ thể, đối với mỗi $u \in U'_{ij}$, $\mathcal{N}_{ij}(u)$ được phân rã thành $\gamma$ khối (ký hiệu là $\mathcal{B}_u^1, \mathcal{B}_u^2, \dots, \mathcal{B}_u^\gamma$) dựa trên khoảng cách từ các tọa độ trong $\mathcal{N}_{ij}(u)$ đến $u$, trong đó $\mathcal{B}_u^l = \{ x \in \mathcal{N}_{ij}(u) : (1+\epsilon_1)^l \leq \delta^2(x, u) < (1+\epsilon_1)^{l+1} \}$. 

% BEGIN PB
\hl{Các ``khối'' này có thể hình dung là các phần tiếp nối giữa hai khối cầu tâm $u$ có bình phương bán kính $R^2$ giữa $(1+\epsilon_1)^l$ và $(1+\epsilon_1)^{l+1}$ trong $\mathbb{R}^d$.

Đó cũng chính là lý do xuất hiện $\log$ trong thuật toán và $l < \gamma$.
}.


% END PB

Sau đó, các khối này được chia tiếp thành hai nhóm dựa trên kích thước: 

$\mathcal{L}(u) = \{ \mathcal{B}_u^l : |\mathcal{B}_u^l| \geq \frac{\epsilon_1^2 \alpha m_i}{(1+\epsilon_1) \log(m_i\Delta_{max}^2)}, l \in [\gamma] \}$ là nhóm các khối lớn và $\mathcal{S}(u) = \{ \mathcal{B}_u^1, \dots, \mathcal{B}_u^\gamma \} \setminus \mathcal{L}(u)$ là nhóm các khối nhỏ. 

\textbf{Sự hội tụ xác suất:}

Mục tiêu là xấp xỉ tốt từng khối lớn trong $\mathcal{L}(u)$ đồng thời cho phép bỏ qua các tọa độ trong các khối nhỏ.

\begin{enumerate}
    \item \textbf{Biến ngẫu nhiên:} Đối với mỗi $p \in S_{ij}$, xét biến ngẫu nhiên Bernoulli cho việc $p$ rơi vào một khối cụ thể.
    \item \textbf{Bất đẳng thức Chernoff:} Với kích thước mẫu $|S_{ij}|$ được chọn, kỳ vọng số điểm rơi vào mỗi khối lớn đủ lớn để xác suất sai lệch quá $\epsilon_1$ lần kỳ vọng bị chặn bởi một hàm mũ âm. Cụ thể, $Pr(|X - \mathbb{E}[X]| \geq \epsilon_1 \mathbb{E}[X]) \leq 2e^{-\epsilon_1^2 \mathbb{E}[X]/3}$.
    \item \textbf{Chặn hợp:} Bằng cách lấy tổng xác suất lỗi trên tất cả các khối và các tọa độ ứng viên, tác giả đảm bảo rằng bộ ước lượng $\omega$ hoạt động chính xác với xác suất cao trên toàn không gian ứng viên.
\end{enumerate}

Với bộ ước lượng đã được chứng minh là hội tụ về giá trị thực, việc tìm $c_{ij}$ tại bước 11 nhanh hơn vì số lượng ứng viên $|U'_{ij}|$ chỉ phụ thuộc logarit vào $\Delta_{max}$ và $m$, trong khi việc tính toán mỗi giá trị $\omega(u)$ chỉ tốn thời gian phụ thuộc vào kích thước mẫu $|S_{ij}|$ thay vì kích thước toàn bộ dữ liệu $m_i$. Cuối cùng, bằng cách sử dụng Bổ đề \ref{lemma:7}, Định lý \ref{thm:2} có thể được chứng minh để độ phức tạp thời gian tuyến tính $O(md) + \tilde{O}(\epsilon^{-5}kd/\alpha)$ cho bài toán có hỗ trợ học.

% L 8
\begin{lemma}
\label{lemma:8}
Giả sử $S_{ij}$ là một mẫu được lấy ngẫu nhiên từ cụm dự đoán $P_{ij}$ với kích thước mẫu $|S_{ij}| = \tilde{O}(1/\alpha \epsilon_1^4)$. Với xác suất ít nhất $1 - \frac{\epsilon_1}{m^3d \log^2(m\Delta_{\max}^2)}$, các bất đẳng thức sau đây đồng thời xảy ra cho mọi khối lớn $\mathcal{B}_u^l \in \mathcal{L}(u)$ và tập các điểm xa nhất $\mathcal{O}(u)$:
\[ (1 - \epsilon_1)\mathbb{E}[|\mathcal{B}_u^l \cap S_{ij}|] \leq |\mathcal{B}_u^l \cap S_{ij}| \leq (1 + \epsilon_1)\mathbb{E}[|\mathcal{B}_u^l \cap S_{ij}|] \]
\[ (1 - \epsilon_1)\mathbb{E}[|\mathcal{O}(u) \cap S_{ij}|] \leq |\mathcal{O}(u) \cap S_{ij}| \leq (1 + \epsilon_1)\mathbb{E}[|\mathcal{O}(u) \cap S_{ij}|] \]
\end{lemma}



% L 9

\begin{lemma}
\label{lemma:9}
Gọi $\mathcal{J}(u)$ là tập hợp các tọa độ nằm trong các khối nhỏ đối với một tọa độ ứng viên $u$. Với xác suất ít nhất $1 - \frac{\epsilon_1}{m^3d \log^2(m\Delta_{\max}^2)}$, giao của tập mẫu $S_{ij}$ và $\mathcal{J}(u)$ bị chặn như sau:
\[ |\mathcal{J}(u) \cap S_{ij}| \leq 2\epsilon_1\alpha|S_{ij}| \]
\end{lemma}



% L 10

\begin{lemma}
\label{lemma:10}
Cho một tọa độ ứng viên bất kỳ $u \in U'_{ij}$. Với xác suất cao (xác suất hằng số), ước lượng $\omega(u)$ thỏa mãn các chặn sau:
\[ 
\frac{\delta^2(P_{ij} \setminus \mathcal{F}^\dagger(u), u)}{1 + 7\epsilon_1} \leq \omega(u) \leq (1 + \epsilon_1)^2 \delta^2(\mathcal{N}_{ij}(u), u) 
\]
trong đó:
\begin{itemize}
    \item $\mathcal{F}^\dagger(u)$ là tập hợp gồm $(2 + 20\epsilon_1)\alpha m_i$ tọa độ xa nhất từ $P_{ij}$ đến $u$.
    \item $\mathcal{N}_{ij}(u)$ là tập hợp gồm $(1-\alpha)m_i$ tọa độ gần nhất trong $P_{ij}$ đến $u$.
\end{itemize}
\end{lemma}



% L 11

\begin{lemma}
\label{lemma:11}
Với tập hợp các tọa độ $I_{ij}$ được xác định bởi thuật toán Fast-Estimation, chặn sau đây luôn thỏa mãn:
\[ \delta^2(\overline{I_{ij}}, \overline{Q_{ij}}) \leq \frac{13\alpha - 15\alpha^2}{(1 - 3\alpha - \epsilon)(1 - 2\alpha - \epsilon)} \frac{\delta^2(Q_{ij}, \overline{Q_{ij}})}{|Q_{ij}|} \]
\end{lemma}



% T 2

\begin{theorem}
\label{thm:2}
Thuật toán Fast-Estimation xấp xỉ $(1 + O(\alpha))$ cho bài toán k-means có hỗ trợ học (learning-augmented) trong thời gian $O(md) + \tilde{O}(\epsilon^{-5}kd/\alpha)$ với xác suất hằng số, với tỷ lệ lỗi nhãn $\alpha \in (0, 1/3 - \epsilon)$. 
\end{theorem}
